\documentclass[12pt]{report}
\usepackage[utf8]{inputenc}
\usepackage[russian]{babel}
%\usepackage[14pt]{extsizes}
\usepackage{listings}
\usepackage{graphicx}
\usepackage{amsmath,amsfonts,amssymb,amsthm,mathtools} 
\usepackage{pgfplots}
\usepackage{filecontents}
\usepackage{float}
\usepackage{indentfirst}
\usepackage{eucal}
\usepackage{enumitem}
%s\documentclass[openany]{book}
\frenchspacing

\usepackage{indentfirst} % Красная строка

\usetikzlibrary{datavisualization}
\usetikzlibrary{datavisualization.formats.functions}

\usepackage{amsmath}


% Для листинга кода:
\lstset{ %
	language=c,                 % выбор языка для подсветки (здесь это С)
	basicstyle=\small\sffamily, % размер и начертание шрифта для подсветки кода
	numbers=left,               % где поставить нумерацию строк (слева\справа)
	numberstyle=\tiny,           % размер шрифта для номеров строк
	stepnumber=1,                   % размер шага между двумя номерами строк
	numbersep=5pt,                % как далеко отстоят номера строк от подсвечиваемого кода
	showspaces=false,            % показывать или нет пробелы специальными отступами
	showstringspaces=false,      % показывать или нет пробелы в строках
	showtabs=false,             % показывать или нет табуляцию в строках
	frame=single,              % рисовать рамку вокруг кода
	tabsize=2,                 % размер табуляции по умолчанию равен 2 пробелам
	captionpos=t,              % позиция заголовка вверху [t] или внизу [b] 
	breaklines=true,           % автоматически переносить строки (да\нет)
	breakatwhitespace=false, % переносить строки только если есть пробел
	escapeinside={\#*}{*)}   % если нужно добавить комментарии в коде
}


\usepackage[left=2cm,right=2cm, top=2cm,bottom=2cm,bindingoffset=0cm]{geometry}
% Для измененных титулов глав:
\usepackage{titlesec, blindtext, color} % подключаем нужные пакеты
\definecolor{gray75}{gray}{0.75} % определяем цвет
\newcommand{\hsp}{\hspace{20pt}} % длина линии в 20pt
% titleformat определяет стиль
\titleformat{\chapter}[hang]{\Huge\bfseries}{\thechapter\hsp\textcolor{gray75}{|}\hsp}{0pt}{\Huge\bfseries}


% plot
\usepackage{pgfplots}
\usepackage{filecontents}
\usetikzlibrary{datavisualization}
\usetikzlibrary{datavisualization.formats.functions}

\begin{document}
	%\def\chaptername{} % убирает "Глава"
	\thispagestyle{empty}
	\begin{titlepage}
		\noindent \begin{minipage}{0.15\textwidth}
			\includegraphics[width=\linewidth]{b_logo}
		\end{minipage}
		\noindent\begin{minipage}{0.9\textwidth}\centering
			\textbf{Министерство науки и высшего образования Российской Федерации}\\
			\textbf{Федеральное государственное бюджетное образовательное учреждение высшего образования}\\
			\textbf{~~~«Московский государственный технический университет имени Н.Э.~Баумана}\\
			\textbf{(национальный исследовательский университет)»}\\
			\textbf{(МГТУ им. Н.Э.~Баумана)}
		\end{minipage}
		
		\noindent\rule{18cm}{3pt}
		\newline\newline
		\noindent ФАКУЛЬТЕТ $\underline{\text{«Информатика и системы управления»}}$ \newline\newline
		\noindent КАФЕДРА $\underline{\text{«Программное обеспечение ЭВМ и информационные технологии»}}$\newline\newline\newline\newline\newline
		
		\begin{center}
			\noindent\begin{minipage}{1.1\textwidth}\centering
				\Large\textbf{  Отчет по лабораторной работе №7}\newline
				\textbf{по дисциплине <<Функциональное и логическое}\newline
				\textbf{~~~программирование>>}\newline\newline
			\end{minipage}
		\end{center}
		
		\noindent\textbf{Тема} $\underline{\text{Рекурсивные функции~~~~~~~~~~~~~~~~~~~~~~~~~~~~~~~~~~~~~~~~~~~~~~~~~~~~}}$\newline\newline
		\noindent\textbf{Студент} $\underline{\text{Варламова Е. А.~~~~~~~~~~~~~~~~~~~~~~~~~~~~~~~~~~~~~~~~~~~~~~~~~~~~~~~}}$\newline\newline
		\noindent\textbf{Группа} $\underline{\text{ИУ7-61Б~~~~~~~~~~~~~~~~~~~~~~~~~~~~~~~~~~~~~~~~~~~~~~~~~~~~~~~~~~~~~~~~~~~~}}$\newline\newline
		\noindent\textbf{Оценка (баллы)} $\underline{\text{~~~~~~~~~~~~~~~~~~~~~~~~~~~~~~~~~~~~~~~~~~~~~~~~~~~~~~~~~~~~~~~~~~~}}$\newline\newline
		\noindent\textbf{Преподаватель} $\underline{\text{Толпинская Н.Б., Строганов Ю. В.~~~~~~~~~~~~~~~~~~~~}}$\newline\newline\newline
		
		\begin{center}
			\vfill
			Москва~---~\the\year
			~г.
		\end{center}
	\end{titlepage}
\setcounter{page}{2}
\section*{Задание 1}
\subsection*{Постановка задачи}
Написать хвостовую рекурсивную функцию my-reverse, которая развернет верхний
уровень своего списка-аргумента lst.

\begin{lstlisting}
(defun move (lst res)
(cond ((null lst) res)
      (t (move (cdr lst) (cons (car lst) res) ))
) )

(defun my_reverse (lst)
(move lst ()))
\end{lstlisting}

\section*{Задание №2}
Написать функцию, которая возвращает первый элемент списка-аргумента, который сам
является непустым списком.

\begin{lstlisting}
(defun flist (lst)
(cond 
( (and (listp (car lst) ) (not (null (car lst)))) (car lst))
((cdr lst) (flist (cdr lst)))
)) 
\end{lstlisting}

\section*{Задание №3}
Написать функцию, которая выбирает из заданного списка только те числа, которые
больше 1 и меньше 10.

(Вариант: между двумя заданными границами. )
\subsection*{Решение}
\begin{lstlisting}[label=third,caption=Решение задания №3, language=lisp]
(defun check_borders (x1 x2 el) (and (> el x1) (< el x2)))

(defun select_between (lst x1 x2)
(cond
((null lst) NIL)
( (check_borders x1 x2 (car lst)) (cons (car lst) (select_between (cdr lst) x1 x2)) )
(T (select_between (cdr lst) x1 x2))))
\end{lstlisting}

\section*{Задание №4}
Напишите рекурсивную функцию, которая умножает на заданное число-аргумент все
числа из заданного списка-аргумента.

\subsection*{Решение}
Все элементы списка --- числа.
\begin{lstlisting}[label=third,caption=Решение задания №3, language=lisp]
(defun prod (lst num) (cond ((null lst) Nil)
						    (T (nconc (list (* (car lst) num)) (prod (cdr lst) num)))))
\end{lstlisting}

Элементы списка -- любые объекты.
\begin{lstlisting}[label=third,caption=Решение задания №3, language=lisp]
(defun prod (lst num) (cond ((null lst) Nil)
							((listp (car lst)) (nconc (list (prod (car lst) num)) (prod (cdr lst) num)))
						    (T (nconc (list (* (car lst) num)) (prod (cdr lst) num)))))
\end{lstlisting}

\section*{Задание №5}
Напишите функцию, select-between, которая из списка-аргумента, содержащего только
числа, выбирает только те, которые расположены между двумя указанными границами-
аргументами и возвращает их в виде списка (упорядоченного по возрастанию списка чисел
(+ 2 балла)).

\subsection*{Решение}
\begin{lstlisting}[label=5,caption=Решение задания №5, language=lisp]
(defun check_borders (x1 x2 el) (and (> el x1) (< el x2)))

(defun select_between (lst x1 x2)
(cond
((null lst) NIL)
( (check_borders x1 x2 (car lst)) (cons (car lst) (select_between (cdr lst) x1 x2)) )
(T (select_between (cdr lst) x1 x2))))

\end{lstlisting}


\section*{Задание №6}
Написать рекурсивную версию (с именем rec-add) вычисления суммы чисел заданного
списка.
\subsection*{Решение}

Одноуровневый смешанный список.
\begin{lstlisting}[label=5,caption=Решение задания №6а, language=lisp]

(defun add (lst acc)
(cond
( (null lst) acc)
( (numberp (car lst)) (add (cdr lst) (+ acc (car lst))))
( T (add (cdr lst) acc))
))
(defun rec_add (lst) (add lst 0))

\end{lstlisting}

Структурированный список.

\begin{lstlisting}[label=5,caption=Решение задания №6б, language=lisp]

(defun add (lst acc)
(cond
( (null lst) acc)
( (numberp (car lst)) (add (cdr lst) (+ acc (car lst))))
( (listp (car lst)) 
(let ((sum (rec_add (car lst))))
(add (cdr lst) (+ acc sum))
))
(T (add (cdr lst) acc))
))

(defun rec-add (lst) (add lst 0))

\end{lstlisting}

\section*{Задание №7}
Написать рекурсивную версию с именем recnth функции nth.
\begin{lstlisting}[label=5,caption=Решение задания №7, language=lisp]
(defun rnth (lst n ind)
(cond ((= ind n) (car lst))
      (T (rnth (cdr lst) n (1+ ind)))
))
(defun recnth (n lst)
(rnth lst n 0))

\end{lstlisting}

\section*{Задание №8}
Написать рекурсивную функцию allodd, которая возвращает t когда все элементы списка нечетные.
\subsection*{Решение}
\begin{lstlisting}[label=5,caption=Решение задания №8, language=lisp]
(defun allodd (lst)
(cond ((null lst) T)
((oddp (car lst)) (allodd (cdr lst)))
))
\end{lstlisting}	

\section*{Задание №9}
Написать рекурсивную функцию, которая возвращает первое нечетное число из списка
(структурированного), возможно создавая некоторые вспомогательные функции.
\subsection*{Решение}
\begin{lstlisting}[label=5,caption=Решение задания №9, language=lisp]
(defun f (lst)
(cond 
((null lst) nil)
( (and (numberp (car lst)) (oddp (car lst))) (car lst))
( (listp (car lst)) (or (f (car lst)) (f (cdr lst)) ))
( (f (cdr lst)))
))
\end{lstlisting}	

\section*{Задание №10}
Используя cons-дополняемую рекурсию с одним тестом завершения,
написать функцию которая получает как аргумент список чисел, а возвращает список
квадратов этих чисел в том же порядке
\subsection*{Решение}
\begin{lstlisting}[label=5,caption=Решение задания №10, language=lisp]
(defun sqr (lst)
(cond 
( (cdr lst) (cons (* (car lst) (car lst)) (sqr (cdr lst))))
( T (list (* (car lst) (car lst)))
))
\end{lstlisting}	
	
\end{document}