\documentclass[12pt]{report}
\usepackage[utf8]{inputenc}
\usepackage[russian]{babel}
%\usepackage[14pt]{extsizes}
\usepackage{listings}
\usepackage{graphicx}
\usepackage{amsmath,amsfonts,amssymb,amsthm,mathtools} 
\usepackage{pgfplots}
\usepackage{filecontents}
\usepackage{float}
\usepackage{indentfirst}
\usepackage{eucal}
\usepackage{enumitem}
%s\documentclass[openany]{book}
\frenchspacing

\usepackage{indentfirst} % Красная строка

\usetikzlibrary{datavisualization}
\usetikzlibrary{datavisualization.formats.functions}

\usepackage{amsmath}


% Для листинга кода:
\lstset{ %
	language=c,                 % выбор языка для подсветки (здесь это С)
	basicstyle=\small\sffamily, % размер и начертание шрифта для подсветки кода
	numbers=left,               % где поставить нумерацию строк (слева\справа)
	numberstyle=\tiny,           % размер шрифта для номеров строк
	stepnumber=1,                   % размер шага между двумя номерами строк
	numbersep=5pt,                % как далеко отстоят номера строк от подсвечиваемого кода
	showspaces=false,            % показывать или нет пробелы специальными отступами
	showstringspaces=false,      % показывать или нет пробелы в строках
	showtabs=false,             % показывать или нет табуляцию в строках
	frame=single,              % рисовать рамку вокруг кода
	tabsize=2,                 % размер табуляции по умолчанию равен 2 пробелам
	captionpos=t,              % позиция заголовка вверху [t] или внизу [b] 
	breaklines=true,           % автоматически переносить строки (да\нет)
	breakatwhitespace=false, % переносить строки только если есть пробел
	escapeinside={\#*}{*)}   % если нужно добавить комментарии в коде
}


\usepackage[left=2cm,right=2cm, top=2cm,bottom=2cm,bindingoffset=0cm]{geometry}
% Для измененных титулов глав:
\usepackage{titlesec, blindtext, color} % подключаем нужные пакеты
\definecolor{gray75}{gray}{0.75} % определяем цвет
\newcommand{\hsp}{\hspace{20pt}} % длина линии в 20pt
% titleformat определяет стиль
\titleformat{\chapter}[hang]{\Huge\bfseries}{\thechapter\hsp\textcolor{gray75}{|}\hsp}{0pt}{\Huge\bfseries}


% plot
\usepackage{pgfplots}
\usepackage{filecontents}
\usetikzlibrary{datavisualization}
\usetikzlibrary{datavisualization.formats.functions}

\begin{document}
	%\def\chaptername{} % убирает "Глава"
	\thispagestyle{empty}
	\begin{titlepage}
		\noindent \begin{minipage}{0.15\textwidth}
			\includegraphics[width=\linewidth]{b_logo}
		\end{minipage}
		\noindent\begin{minipage}{0.9\textwidth}\centering
			\textbf{Министерство науки и высшего образования Российской Федерации}\\
			\textbf{Федеральное государственное бюджетное образовательное учреждение высшего образования}\\
			\textbf{~~~«Московский государственный технический университет имени Н.Э.~Баумана}\\
			\textbf{(национальный исследовательский университет)»}\\
			\textbf{(МГТУ им. Н.Э.~Баумана)}
		\end{minipage}
		
		\noindent\rule{18cm}{3pt}
		\newline\newline
		\noindent ФАКУЛЬТЕТ $\underline{\text{«Информатика и системы управления»}}$ \newline\newline
		\noindent КАФЕДРА $\underline{\text{«Программное обеспечение ЭВМ и информационные технологии»}}$\newline\newline\newline\newline\newline
		
		\begin{center}
			\noindent\begin{minipage}{1.1\textwidth}\centering
				\Large\textbf{  Отчет по лабораторной работе №4}\newline
				\textbf{по дисциплине <<Функциональное и логическое}\newline
				\textbf{~~~программирование>>}\newline\newline
			\end{minipage}
		\end{center}
		
		\noindent\textbf{Тема} $\underline{\text{Использование управляющих структур, работа со списками}}$\newline\newline
		\noindent\textbf{Студент} $\underline{\text{Варламова Е. А.~~~~~~~~~~~~~~~~~~~~~~~~~~~~~~~~~~~~~~~~~~~~~~~~~~~~~~~~~~~~}}$\newline\newline
		\noindent\textbf{Группа} $\underline{\text{ИУ7-61Б~~~~~~~~~~~~~~~~~~~~~~~~~~~~~~~~~~~~~~~~~~~~~~~~~~~~~~~~~~~~~~~~~~~~}}$\newline\newline
		\noindent\textbf{Оценка (баллы)} $\underline{\text{~~~~~~~~~~~~~~~~~~~~~~~~~~~~~~~~~~~~~~~~~~~~~~~~~~~~~~~~~~~~~~~~~~~}}$\newline\newline
		\noindent\textbf{Преподаватель} $\underline{\text{Толпинская Н.Б., Строганов Ю. В.~~~~~~~~~~~~~~~~~~~~}}$\newline\newline\newline
		
		\begin{center}
			\vfill
			Москва~---~\the\year
			~г.
		\end{center}
	\end{titlepage}
\setcounter{page}{2}
\section*{Задание 1}
\subsection*{Постановка задачи}
Чем принципиально отличаются функции \texttt{cons}, \texttt{list}, \texttt{append}?\\
\indent Пусть \texttt{(setf lst1 '(a b))} \texttt{(setf lst2 '(c d))}\\
\indent Каковы результаты следующих выражений?

\begin{lstlisting}
(cons lst1 lst2) -> ((A B) C D)
(list lst1 lst2) -> ((A B) (C D))
(append lst1 lst2) -> (A B C D)
\end{lstlisting}

\section*{Задание №2}
Каковы результаты вычисления следующих выражений?

\begin{lstlisting}
(reverse |()) -> рекурсия
(last ()) -> Nil
(reverse '(a)) -> (a)
(last '(a)) -> (a)
(reverse '((a b c))) -> ((a b c))
(last '((a b c))) -> ((a b c))

\end{lstlisting}


\section*{Задание №3}
Написать, по крайней мере, два варианта функции, которая возвращает последний элемент своего списка-аргумента.
\subsection*{Решение}
\begin{lstlisting}[label=third,caption=Решение задания №3, language=lisp]
(defun lastlast (lst)
	(if (cdr lst)
		(lastlast (cdr lst))
		(car lst))
)

(defun lastlast (lst)
	(car(reverse lst))
)
\end{lstlisting}

\section*{Задание №4}
Написать, по крайней мере, два варианта функции, которая возвращает свой список-аргумент без последнего элемента.

\subsection*{Решение}
\begin{lstlisting}[label=third,caption=Решение задания №4, language=lisp]
(defun withount_last (lst) 
(if (cdr lst)
    (cons (car lst) (withount_last (cdr lst)))
))

(defun withount_last (lst)
(reverse (cdr(reverse lst)))
)
\end{lstlisting}

\section*{Задание №5}
Написать простой вариант игры в кости, в котором бросаются две правильные кости. Если сумма выпавших очков равна 7 или 11 -- выигрыш, если выпало $(1,1)$ или $(6,6)$ --- игрок право снова бросить кости, во всех остальных случаях ход переходит ко второму игроку, но запоминается сумма выпавших очков. Если второй игрок не выигрывает абсолютно, то выигрывает тот игрок, у которого больше очков. Результат игры и значения выпавших костей выводить на экран с помощью функции \texttt{print}.

\subsection*{Решение}
\begin{lstlisting}[label=5,caption=Решение задания №5, language=lisp]
(defun get_points () (list (random 7) (random 7)))

(defun is_check_pair (pair check_pair)
(and (not (atom pair)) (equal (car pair) (car check_pair)) (equal (cadr pair) (cadr check_pair))
))

(defun sum_pair (pair)
(and (not (atom pair)) (+ (car pair) (cadr pair)))
)

(defun check_absolute_win(pair)
(and (not (atom pair))
(or
    (equal (sum_pair pair) 7)
    (equal (sum_pair pair) 11)
)
))

(defun can_rerun (pair)
    (or (is_check_pair pair '(1 1)) (is_check_pair pair '(6 6)))
)

(defun logic (points)
(if (check_absolute_win points)
    'won
    (if (can_rerun points)
    ((lambda ()
        (princ "Do you want rerun?[y if yes, any if no]")
        (terpri)
        (if (equal (read) 'y)
            (setf p1 (get_points))
        )
    ))
    )
))

(defun winner (p1 p2)
(if (> (sum_pair p1) (sum_pair p2))
    (princ "First player won!")
    (princ "Second player won!")
))

(defun game ()
        (setf p1 (get_points))
        (princ "Game started!")
        (terpri)
        (princ "First player's turn. Points: ")
        (princ p1)
        (terpri)
        (setf res1 (logic p1))
        (if (equal res1 'won)
            (princ "First player won!")
            ((lambda ()
                (if (not (equal res1 nil))
                    ((lambda ()
                    (setf p1 res1)
                    (princ "First player's points: ")
                    (princ p1)
                    (terpri)
                    ))
                 )
                (setf p2 (get_points))
                (princ "Second player's turn. Points: ")
                (princ p2)
                (terpri)
                (setf res2 (logic p2))
                (if (equal res2 'won)
                    (princ "Second player won!")
                    ((lambda ()
                        (if (not (equal res2 nil))
                            ((lambda ()
                            (setf p2 res2)
                            (princ "Second player's points: ")
                            (princ p2)
                            (terpri)
                            ))
                        )
                        (winner p1 p2)
                    ))
                )
            ))
        )
)
\end{lstlisting}


\section*{Контрольные вопросы}
\textbf{Вопрос 1.} Синтаксическая форма и хранение программы в памяти.

\textbf{Ответ.}
В Lisp формы представления программы и обрабатываемых ею данных одинаковы – они представлены в виде S-выражений. Программы могут обрабатывать и преобразовывать другие программы или сами себя. В памяти программа представляется в виде бинарных узлов, так как она состоит из S-выражений.\newline

\textbf{Вопрос 2.} Трактовка элементов списка. 

\textbf{Ответ.}
Если отсутствует блокировка вычислений, то первый элемент списка трактуется как имя функции, а остальные элементы – как аргументы функции.\newline
	
\textbf{Вопрос 3.} Порядок реализации программы.

\textbf{Ответ.}
Работа программы циклична: сначала программа ожидает ввода S-выражения, затем передает полученное S-выражение интерпретатору – функции eval, а в конце, после отработки функции eval, выводит последний полученный результат.\newline

\textbf{Вопрос 4.} Способы определения функций\\
\indent\textbf{Ответ.} Существует два способа определений функций:

\begin{itemize}
	\item через \texttt{defun};
	\item через \texttt{lambda}.
\end{itemize}

Пример \texttt{defun}:
\begin{lstlisting}
(defun func-name (args-list) function-body)
(defun get-cube(y) (* y y y))
(get-cube y)
\end{lstlisting}

Пример \texttt{lambda}:
\begin{lstlisting}
(lambda (args-list) function-body)
((lambda (x) (* x x)) 2)
\end{lstlisting} ""
	
	
\end{document}