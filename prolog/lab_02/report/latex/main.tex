\documentclass[12pt]{report}
\usepackage[utf8]{inputenc}
\usepackage[russian]{babel}
%\usepackage[14pt]{extsizes}
\usepackage{listings}
\usepackage{graphicx}
\usepackage{amsmath,amsfonts,amssymb,amsthm,mathtools} 
\usepackage{pgfplots}
\usepackage{filecontents}
\usepackage{float}
\usepackage{comment}
\usepackage{indentfirst}
\usepackage{eucal}
\usepackage{enumitem}
%s\documentclass[openany]{book}
\frenchspacing

\usepackage{indentfirst} % Красная строка

\usetikzlibrary{datavisualization}
\usetikzlibrary{datavisualization.formats.functions}

\usepackage{amsmath}


% Для листинга кода:
\lstset{ %
	language=c,                 % выбор языка для подсветки (здесь это С)
	basicstyle=\small\sffamily, % размер и начертание шрифта для подсветки кода
	numbers=left,               % где поставить нумерацию строк (слева\справа)
	numberstyle=\tiny,           % размер шрифта для номеров строк
	stepnumber=1,                   % размер шага между двумя номерами строк
	numbersep=5pt,                % как далеко отстоят номера строк от подсвечиваемого кода
	showspaces=false,            % показывать или нет пробелы специальными отступами
	showstringspaces=false,      % показывать или нет пробелы в строках
	showtabs=false,             % показывать или нет табуляцию в строках
	frame=single,              % рисовать рамку вокруг кода
	tabsize=2,                 % размер табуляции по умолчанию равен 2 пробелам
	captionpos=t,              % позиция заголовка вверху [t] или внизу [b] 
	breaklines=true,           % автоматически переносить строки (да\нет)
	breakatwhitespace=false, % переносить строки только если есть пробел
	escapeinside={\#*}{*)}   % если нужно добавить комментарии в коде
}


\usepackage[left=2cm,right=2cm, top=2cm,bottom=2cm,bindingoffset=0cm]{geometry}
% Для измененных титулов глав:
\usepackage{titlesec, blindtext, color} % подключаем нужные пакеты
\definecolor{gray75}{gray}{0.75} % определяем цвет
\newcommand{\hsp}{\hspace{20pt}} % длина линии в 20pt
% titleformat определяет стиль
\titleformat{\section}[hang]{\Huge\bfseries}{\thechapter\hsp\textcolor{gray75}{|}\hsp}{0pt}{\Huge\bfseries}


% plot
\usepackage{pgfplots}
\usepackage{filecontents}
\usetikzlibrary{datavisualization}
\usetikzlibrary{datavisualization.formats.functions}

\begin{document}
	%\def\sectionname{} % убирает "Глава"
	\thispagestyle{empty}
	\begin{titlepage}
		\noindent \begin{minipage}{0.15\textwidth}
			\includegraphics[width=\linewidth]{b_logo}
		\end{minipage}
		\noindent\begin{minipage}{0.9\textwidth}\centering
			\textbf{Министерство науки и высшего образования Российской Федерации}\\
			\textbf{Федеральное государственное бюджетное образовательное учреждение высшего образования}\\
			\textbf{~~~«Московский государственный технический университет имени Н.Э.~Баумана}\\
			\textbf{(национальный исследовательский университет)»}\\
			\textbf{(МГТУ им. Н.Э.~Баумана)}
		\end{minipage}
		
		\noindent\rule{18cm}{3pt}
		\newline\newline
		\noindent ФАКУЛЬТЕТ $\underline{\text{«Информатика и системы управления»}}$ \newline\newline
		\noindent КАФЕДРА $\underline{\text{«Программное обеспечение ЭВМ и информационные технологии»}}$\newline\newline\newline\newline\newline
		
		\begin{center}
			\noindent\begin{minipage}{1.1\textwidth}\centering
				\Large\textbf{Отчет по лабораторной работе №12}\newline
				\textbf{по дисциплине <<Функциональное и логическое}\newline
				\textbf{~~~программирование>>}\newline\newline
			\end{minipage}
		\end{center}
		
		\noindent\textbf{Тема} $\underline{\text{Среда Visual Prolog. Структура программы. Работа программы}}$\newline\newline
		\noindent\textbf{Студент} $\underline{\text{Варламова Е. А.~~~~~~~~~~~~~~~~~~~~~~~~~~~~~~~~~~~~~~~~~~~~~~~~~~~~~~~~~~~~~~~~~}}$\newline\newline
		\noindent\textbf{Группа} $\underline{\text{ИУ7-61Б~~~~~~~~~~~~~~~~~~~~~~~~~~~~~~~~~~~~~~~~~~~~~~~~~~~~~~~~~~~~~~~~~~~~~~~~~}}$\newline\newline
		\noindent\textbf{Оценка (баллы)} $\underline{\text{~~~~~~~~~~~~~~~~~~~~~~~~~~~~~~~~~~~~~~~~~~~~~~~~~~~~~~~~~~~~~~~~~~~~~~~~}}$\newline\newline
		\noindent\textbf{Преподаватель} $\underline{\text{Толпинская Н.Б., Строганов Ю. В.~~~~~~~~~~~~~~~~~~~~~~~~~~}}$\newline\newline\newline
		
		\begin{center}
			\vfill
			Москва~---~\the\year
			~г.
		\end{center}
	\end{titlepage}
\section*{Задание}
Составить программу, т.е. модель предметной области – базу знаний, объединив в ней информацию – знания:
\begin{itemize}
    \item «Телефонный справочник»: Фамилия, Noтел, Адрес – структура (Город, Улица, Noдома, Noкв),
    \item «Автомобили»: Фамилия\_владельца, Марка, Цвет, Стоимость, и др.,
    \item «Вкладчики банков»: Фамилия, Банк, счет, сумма, др.
\end{itemize}
Владелец может иметь несколько телефонов, автомобилей, вкладов (Факты). Используя правила, обеспечить возможность поиска:
\begin{enumerate}

\item По No телефона найти: Фамилию, Марку автомобиля, Стоимость автомобиля (может быть несколько),
\item Используя сформированное в пункте а) правило, по No телефона найти: только Марку автомобиля (автомобилей может быть несколько),
\item Используя простой, не составной вопрос: по Фамилии (уникальна в городе, но в разных городах есть однофамильцы) и Городу проживания найти: Улицу проживания, Банки, в которых есть вклады и Noтелефона.
\end{enumerate}
И для 1., и для 2. описать словесно порядок поиска ответа на вопрос, указав, как выбираются знания, и, при этом, для каждого этапа унификации, выписать подстановку – наибольший общий унификатор, и соответствующие примеры термов.
\chapter*{Решение}
\begin{lstlisting}
domains
name, phone, univer = symbol.
model, color = symbol.
bankName = symbol.
amount, price = integer.
city, street, house, flat = symbol.
address = address(city, street, house, flat).

predicates

tel(name, phone, address).
car(name, model, color, price).
deposit(name, bankName, amount).
fs(name, model, price, phone).

clauses

%fs(Name, Model, Price, Phone):- tel(Name, Phone, _), car(Name, Model, _, Price).
%ss(Name, City, Street, Banks, Phone):- tel(Name, Phone, address(City, Street, _, _)), deposit(Name, Banks, _).

tel(anton, "812314214", address("moscow", "olenevaya", "12", "4")).
tel(egor, "814314214", address("moscow", "olenevaya2", "12", "4")).
tel(darya, "817314214", address("moscow", "olenevaya", "13", "4")).
tel(valera, "816314214", address("moscow", "olenevaya2", "16", "4")).
tel(denis, "815314214", address("moscow", "olenevaya3", "12", "4")).

car(anton, "x6", "red", 10000000).
car(darya, "x1", "yellow", 15000000).
car(denis, "juke", "red", 15000000).

deposit(egor, "sber", 1000).
deposit(anton, "tinkoff", 20000).
deposit(denis, "raif", 100000).
deposit(valera, "sber", 10000).

goal
%fs(X, Y, Z, "815314214").

%fs(_, Y, _, "815314214").

%ss(egor, "moscow", X, Y, Z).
\end{lstlisting}


Порядок формирования результата для 1-го вопроса:

\begin{table}[H]
	\begin{center}
		\begin{tabular}{|c c c |} 
			\hline
			№ шага & Сравниваемые термы; результаты; подстановка, если есть & Дальнейшие действия \\  
			\hline
			1 & Сравниваются & Прямой ход \\
			  & fs(Name, Model, Price, Phone) & \\
			  & и fs(X, Y, Z, "815314214")  & \\
			  & Подстановка (Phone - "815314214") &\\
			\hline
			2 & Сравниваются tel(Name, "815314214", \_) & Термы не \\
			  & и fs(Name, Model, Price, Phone). & унифицируемы. \\
			  & Они имеют разные функторы. &Переход к следующему \\
			  & & предложению\\
			\hline
			3 & Сравниваются tel(Name, "815314214", \_) & Термы не \\
			  & и tel(anton, "812314214", address("moscow", "olenevaya", "12", "4"))). & унифицируемы. \\
			  & Они имеют разные функторы. & Переход к следующему \\
			  & & предложению\\
			\hline
			4 & Сравниваются tel(Name, "815314214", \_) & Термы не \\
			  & и tel(egor, "814314214", address( & унифицируемы. \\
		      & "moscow", "olenevaya", "13", "4")).  & Переход к следующему \\
			  & Термы не унифицируемы. & предложению. \\
			\hline
			5 & Сравниваются tel(Name, "815314214", \_) & Термы не \\
			  & и tel(darya, "817314214", address( & унифицируемы. \\
		      & "moscow", "olenevaya", "13", "4")).  & Переход к следующему \\
			  & Термы не унифицируемы. & предложению. \\
			\hline
		    6 & Сравниваются tel(Name, "815314214", \_) & Термы не \\
			  & и tel(valera, "816314214", address( & унифицируемы. \\
			  & "moscow", "olenevaya2", "16", "4")).  & Переход к следующему \\
			  & Термы не унифицируемы. & предложению. \\
			\hline
			7 & Сравниваются tel(Name, "815314214", \_) & Прямой ход. \\
			  & и tel(denis, "815314214", address( & Занесение \\
			  & "moscow", "olenevaya3", "12", "4")). Подстановка & Name = "Denis"{} \\
			  & (Name = "Denis", Phone = "815314214"{}, & в ячейку\\
			  & \_ = address("moscow", "olenevaya3", "12", "4"{})) &\\
			\hline
			9 & Сравниваются car("Denis", Model, \_, Price) & Термы не \\
			  & и fs(Name, Model, Price, Phone). & унифицируемы. \\
			  & Они имеют разные функторы. &Переход к следующему \\
			  & & предложению\\
\hline
\end{tabular}
	\end{center}
\end{table}
\begin{table}[H]
	\begin{center}
		\begin{tabular}{|c c c |} 
			\hline
			10-14 & Сравниваются car("Denis", Model, \_, Price) & Термы не \\
			  & и tel(...). & унифицируемы. \\
			  & Они имеют разные функторы. &Переход к следующему \\
			  & & предложению\\
			\hline
			15 & Сравниваются car("Denis", Model, \_, Price) & Термы не \\
			  & и car(anton, "x6", "red", 10000000). & унифицируемы. \\
			  & Термы не унифицируемы. & Переход к следующему \\
			  & & предложению. \\
			\hline
			16 & Сравниваются car("Denis", Model, \_, Price) & Термы не \\
			  & и car(darya, "x1", "yellow", 15000000). & унифицируемы. \\
			  & Термы не унифицируемы. & Переход к следующему \\
			  & & предложению. \\
			\hline
			17 & Сравниваются car("Denis", Model, \_, Price) & Прямой ход. Занесение \\
			  & и car(denis, "juke", "red", 15000000). & Model = "juke"{}, Price = "15000000"{}\\
			  & Подстановка (Model = "juke, \_ = "red",  & в результирующую ячейку. \\
			  & Cost = "1500000")  & \\
			\hline
			18 & Результат: подстановка & \\
			   & (Name = "Denis"{}, Model = "juke"{}, Cost = "15000000"{}) & \\
			\hline
		\end{tabular}
	\end{center}
\end{table}


Порядок формирования результата для 2-го вопроса:

\begin{table}[H]
	\begin{center}
		\begin{tabular}{|c c c |} 
			\hline
			№ шага & Сравниваемые термы; результаты; подстановка, если есть & Дальнейшие действия \\  
			\hline
			1 & Сравниваются & Прямой ход \\
			  & fs(Name, Model, Price, Phone) & \\
			  & и fs(_, Y, _, "815314214")  & \\
			  & Подстановка (Phone - "815314214") &\\
			\hline
			2 & Сравниваются tel(Name, "815314214", \_) & Термы не \\
			  & и fs(Name, Model, Price, Phone). & унифицируемы. \\
			  & Они имеют разные функторы. &Переход к следующему \\
			  & & предложению\\
			\hline
			3 & Сравниваются tel(Name, "815314214", \_) & Термы не \\
			  & и tel(anton, "812314214", address("moscow", "olenevaya", "12", "4"))). & унифицируемы. \\
			  & Они имеют разные функторы. & Переход к следующему \\
			  & & предложению\\
			\hline
			4 & Сравниваются tel(Name, "815314214", \_) & Термы не \\
			  & и tel(egor, "814314214", address( & унифицируемы. \\
		      & "moscow", "olenevaya", "13", "4")).  & Переход к следующему \\
			  & Термы не унифицируемы. & предложению. \\
			\hline
			5 & Сравниваются tel(Name, "815314214", \_) & Термы не \\
			  & и tel(darya, "817314214", address( & унифицируемы. \\
		      & "moscow", "olenevaya", "13", "4")).  & Переход к следующему \\
			  & Термы не унифицируемы. & предложению. \\
			\hline
        \end{tabular}
        	\end{center}
        \end{table}
        \begin{table}[H]
        	\begin{center}
        		\begin{tabular}{|c c c |} 
			\hline
		    6 & Сравниваются tel(Name, "815314214", \_) & Термы не \\
			  & и tel(valera, "816314214", address( & унифицируемы. \\
			  & "moscow", "olenevaya2", "16", "4")).  & Переход к следующему \\
			  & Термы не унифицируемы. & предложению. \\
			\hline
			7 & Сравниваются tel(Name, "815314214", \_) & Прямой ход. \\
			  & и tel(denis, "815314214", address( & Занесение \\
			  & "moscow", "olenevaya3", "12", "4")). Подстановка & Name = "Denis"{} \\
			  & (Name = "Denis", Phone = "815314214"{}, & в ячейку\\
			  & \_ = address("moscow", "olenevaya3", "12", "4"{})) &\\
			\hline
			9 & Сравниваются car("Denis", Model, \_, Price) & Термы не \\
			  & и fs(Name, Model, Price, Phone). & унифицируемы. \\
			  & Они имеют разные функторы. &Переход к следующему \\
			  & & предложению\\
			\hline
			10-14 & Сравниваются car("Denis", Model, \_, Price) & Термы не \\
			  & и tel(...). & унифицируемы. \\
			  & Они имеют разные функторы. &Переход к следующему \\
			  & & предложению\\
			\hline
			15 & Сравниваются car("Denis", Model, \_, Price) & Термы не \\
			  & и car(anton, "x6", "red", 10000000). & унифицируемы. \\
			  & Термы не унифицируемы. & Переход к следующему \\
			  & & предложению. \\
			\hline
			16 & Сравниваются car("Denis", Model, \_, Price) & Термы не \\
			  & и car(darya, "x1", "yellow", 15000000). & унифицируемы. \\
			  & Термы не унифицируемы. & Переход к следующему \\
			  & & предложению. \\
			\hline
			17 & Сравниваются car("Denis", Model, \_, Price) & Прямой ход. Занесение \\
			  & и car(denis, "juke", "red", 15000000). & Model = "juke"{}, Price = "15000000"{}\\
			  & Подстановка (Model = "juke, \_ = "red",  & в результирующую ячейку. \\
			  & Cost = "1500000")  & \\
			\hline
			18 & Результат: подстановка & \\
			   & (Model = "juke"{}) & \\
			\hline
		\end{tabular}
	\end{center}
\end{table}
\end{document}