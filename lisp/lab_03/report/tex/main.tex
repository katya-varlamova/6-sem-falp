\documentclass[12pt]{report}
\usepackage[utf8]{inputenc}
\usepackage[russian]{babel}
%\usepackage[14pt]{extsizes}
\usepackage{listings}
\usepackage{graphicx}
\usepackage{amsmath,amsfonts,amssymb,amsthm,mathtools} 
\usepackage{pgfplots}
\usepackage{filecontents}
\usepackage{float}
\usepackage{indentfirst}
\usepackage{eucal}
\usepackage{enumitem}
%s\documentclass[openany]{book}
\frenchspacing

\usepackage{indentfirst} % Красная строка

\usetikzlibrary{datavisualization}
\usetikzlibrary{datavisualization.formats.functions}

\usepackage{amsmath}


% Для листинга кода:
\lstset{ %
	language=c,                 % выбор языка для подсветки (здесь это С)
	basicstyle=\small\sffamily, % размер и начертание шрифта для подсветки кода
	numbers=left,               % где поставить нумерацию строк (слева\справа)
	numberstyle=\tiny,           % размер шрифта для номеров строк
	stepnumber=1,                   % размер шага между двумя номерами строк
	numbersep=5pt,                % как далеко отстоят номера строк от подсвечиваемого кода
	showspaces=false,            % показывать или нет пробелы специальными отступами
	showstringspaces=false,      % показывать или нет пробелы в строках
	showtabs=false,             % показывать или нет табуляцию в строках
	frame=single,              % рисовать рамку вокруг кода
	tabsize=2,                 % размер табуляции по умолчанию равен 2 пробелам
	captionpos=t,              % позиция заголовка вверху [t] или внизу [b] 
	breaklines=true,           % автоматически переносить строки (да\нет)
	breakatwhitespace=false, % переносить строки только если есть пробел
	escapeinside={\#*}{*)}   % если нужно добавить комментарии в коде
}


\usepackage[left=2cm,right=2cm, top=2cm,bottom=2cm,bindingoffset=0cm]{geometry}
% Для измененных титулов глав:
\usepackage{titlesec, blindtext, color} % подключаем нужные пакеты
\definecolor{gray75}{gray}{0.75} % определяем цвет
\newcommand{\hsp}{\hspace{20pt}} % длина линии в 20pt
% titleformat определяет стиль
\titleformat{\chapter}[hang]{\Huge\bfseries}{\thechapter\hsp\textcolor{gray75}{|}\hsp}{0pt}{\Huge\bfseries}


% plot
\usepackage{pgfplots}
\usepackage{filecontents}
\usetikzlibrary{datavisualization}
\usetikzlibrary{datavisualization.formats.functions}

\begin{document}
	%\def\chaptername{} % убирает "Глава"
	\thispagestyle{empty}
	\begin{titlepage}
		\noindent \begin{minipage}{0.15\textwidth}
			\includegraphics[width=\linewidth]{b_logo}
		\end{minipage}
		\noindent\begin{minipage}{0.9\textwidth}\centering
			\textbf{Министерство науки и высшего образования Российской Федерации}\\
			\textbf{Федеральное государственное бюджетное образовательное учреждение высшего образования}\\
			\textbf{~~~«Московский государственный технический университет имени Н.Э.~Баумана}\\
			\textbf{(национальный исследовательский университет)»}\\
			\textbf{(МГТУ им. Н.Э.~Баумана)}
		\end{minipage}
		
		\noindent\rule{18cm}{3pt}
		\newline\newline
		\noindent ФАКУЛЬТЕТ $\underline{\text{«Информатика и системы управления»}}$ \newline\newline
		\noindent КАФЕДРА $\underline{\text{«Программное обеспечение ЭВМ и информационные технологии»}}$\newline\newline\newline\newline\newline
		
		\begin{center}
			\noindent\begin{minipage}{1.1\textwidth}\centering
				\Large\textbf{  Отчет по лабораторной работе №3}\newline
				\textbf{по дисциплине <<Функциональное и логическое}\newline
				\textbf{~~~программирование>>}\newline\newline
			\end{minipage}
		\end{center}
		
		\noindent\textbf{Тема} $\underline{\text{Работа интерпретатора Lisp~~}}$\newline\newline
		\noindent\textbf{Студент} $\underline{\text{Варламова Е. А.~~~~~~~~~~~~~}}$\newline\newline
		\noindent\textbf{Группа} $\underline{\text{ИУ7-61Б~~~~~~~~~~~~~~~~~~~~~~~~~}}$\newline\newline
		\noindent\textbf{Оценка (баллы)} $\underline{\text{~~~~~~~~~~~~~~~~~~~~~~~~}}$\newline\newline
		\noindent\textbf{Преподаватель} $\underline{\text{Толпинская Н.Б.}}$\newline\newline\newline
		
		\begin{center}
			\vfill
			Москва~---~\the\year
			~г.
		\end{center}
	\end{titlepage}
	
	
	
\section*{Задание 1}
Написать функцию, которая принимает целое число и возвращает первое четное число, не меньшее аргумента.
\subsection*{Решение}

\begin{lstlisting}[label=first,caption=Решение задания №1, language=lisp]
(defun f (a) (+ a (mod a 2)))
or

(defun f (x) (if (evenp x) x (+ x 1)))
\end{lstlisting}

\section*{Задание №2}
Написать функцию, которая принимает число и возвращает число того же знака, но с модулем на 1 больше модуля аргумента.

\subsection*{Решение}

\begin{lstlisting}[label=second,caption=Решение задания №2, language=lisp]
(defun f (x) (if (> x 0) (+ x 1) (- x 1)))
\end{lstlisting}

\section*{Задание №3}
Написать функцию, которая принимает два числа и возвращает список из этих чисел, расположенный по возрастанию.

\subsection*{Решение}
\begin{lstlisting}[label=third,caption=Решение задания №3, language=lisp]
( defun sort_two (f s) (if (> s f) (list f s) (list s f) ))
\end{lstlisting}

\section*{Задание №4}
Написать функцию, которая принимает три числа и возвращает Т только тогда, когда первое число расположено между вторым и третьим.

\subsection*{Решение}
\begin{lstlisting}[label=4,caption=Решение задания №4, language=lisp]
(defun f ( a b c) (or (and (> b a) (< a c)) (and (< b a) (> a c)) ) )
\end{lstlisting}

\section*{Задание №5}
Каков результат вычисления следующих выражений?

\subsection*{Решение}
\begin{lstlisting}[label=5,caption=Решение задания №5, language=lisp]
(and 'fee 'fie 'foe) -> FOE
(or 'fee 'fie 'foe) -> FEE
(or nil 'fie 'foe) -> FIE
(and nil 'fie 'foe) -> NIL
(and (equal 'abc 'abc) 'yes) -> YES
(or (equal 'abc 'abc) 'yes) -> T
\end{lstlisting}

\section*{Задание №6}
Написать предикат, который принимает два числа-аргумента и возвращает Т, если первое число не меньше второго.

\subsection*{Решение}
\begin{lstlisting}[label=6,caption=Решение задания №6, language=lisp]
(defun f (a b ) (not (< a b)))
\end{lstlisting}

\section*{Задание №7}
Какой из следующих двух вариантов предиката ошибочен и почему?

\subsection*{Решение}
\begin{lstlisting}[label=7,caption=Решение задания №7, language=lisp]
(defun pred1 (x) (and (numberp x) (plusp x))); (1)
(defun pred2 (x) (and (plusp x)(numberp x))); (2)
\end{lstlisting}
Второй вариант ошибочен, так как перед применением функции plusp к аргументу необходимо убедиться в том, что аргумент является числом (and вычисляет аргументы до первого NIL, поэтому если в функцию будет передано не число, в первом варианте ошибки не произойдёт, а во втором произойдёт).

\section*{Задание №8}
Решить задачу 4, используя для ее решения конструкции IF, COND, AND/OR.

\subsection*{Решение}
\begin{lstlisting}[label=8,caption=Решение задания №8, language=lisp]
(defun f (a b c)
	(if (> b a) (> a c) (> c a))
	)

(defun f (a b c)
	(cond ( (> b a) (> a c))
	( T (> c a))
))

(defun f (a b c)
	(or (and (> b a) (< a c)) (and (< b a) (> a c)) )
	)
\end{lstlisting}


\section*{Задание №9}
Переписать функцию how-alike, приведенную в лекции и использующую COND, используя только конструкции IF, AND/OR.

\subsection*{Решение}
\begin{lstlisting}[label=9,caption=Решение задания №9, language=lisp]
(defun how_alike(x y)
(cond ((or (= x y) (equal x y)) 'the_same)
((and (oddp x) (oddp y)) 'both_odd) 
((and (evenp x) (evenp y)) 'both_even) 
(t 'difference) ))

(defun how-alike-if (x y)
	(if (if (= x y) (equal x y)) `the_same (
		if (if (oddp x) (oddp y)) `both_odd (
			if (if (evenp x) (evenp y)) `both_even `difference))))

(defun how-alike-andor (x y)
	(or (and (or (= x y) (equal x y)) `the_same)
	(and (oddp x) (oddp y) `both_odd)
	(and (evenp x) (evenp y) `both_even)
	`difference))
\end{lstlisting}


	
\section*{Контрольные вопросы}
\textbf{Вопрос 1.} Базис языка Lisp. \newline
\indent\textbf{Ответ. }
Базис языка образуют атомы, структуры (точечные пары и списки), базовые функции, базовые функционалы (функции, аргументами и значением которых являются функции).

\textbf{Вопрос 2.} Классификация функций языка Lisp.
	
\textbf{Ответ.} 
	
\begin{itemize}
	\item чистые (с фиксированным количеством аргументов) математические функции;
	\item рекурсивные функции;
	\item специальные функции – формы (принимают произвольное количество аргументов);
	\item псевдофункции (создающие «эффект», например, на экране);
	\item функционалы.
\end{itemize}

\textbf{Вопрос 3.} Способы создания функций.
	
\textbf{Ответ.} 
Функцию можно определить с помощью \textbf{defun} или \textbf{lambda.} (defun имя\_функции (список\_аргументов) тело\_функции).

\textbf{Вопрос 4.} Работа функций Cond, if, and/or. \newline
\indent\textbf{Ответ. }
Сигнатура функции \textbf{cond}:

\indent(cond (предикат-1 результат-1)) \newline
\indent(предикат-2 результат-2) \newline
\indent...\newline
\indent(предикат-n результат-n)\newline

\indent Работа функции \textbf{cond}: 

сначала просматриваются все предикаты в порядке следования, и если хоть один из них истинный, то cond возвращает результат, связанный с этим предикатом. Если ни один предикат не был истинным, то она вернет Nil. 

Сигнатура функции \textbf{if}: 

(if условие выражение-1 выражение-2)\newline

\indent Работа функции \textbf{if}: 

если условие истинно (T), то выполняется выражение-1, иначе (Nil) – выражение-2\newline

Сигнатура функции \textbf{and}: 

(and выражение-1 выражение-2 ... выражение-n)\newline

\indent Работа функции \textbf{and}: 

результат функции будет истинным, если все ее выражения истинны. В таком случае в качестве результата вернется значение выражения-n. В случае, если хотя бы одно выражение ложно (Nil), вычисление последующих выражений не производится и результатом функции является Nil.\newline

Сигнатура функции \textbf{or}: 

(or выражение-1 выражение-2 ... выражение-n)\newline

Работа функции \textbf{or}: 

результат функции будет ложным (Nil), если все ее выражения ложны. В случае, если хотя бы одно выражение истинно, вычисление последующих выражений не производится и результатом функции является значения выражения, которое первым в списке аргументов дало в результате истину.\newline
	
\bibliographystyle{utf8gost705u}  % стилевой файл для оформления по ГОСТу
	
\bibliography{51-biblio}          % имя библиографической базы (bib-файла)
	
	
\end{document}