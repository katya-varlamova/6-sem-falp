\documentclass[12pt]{report}
\usepackage[utf8]{inputenc}
\usepackage[russian]{babel}
%\usepackage[14pt]{extsizes}
\usepackage{listings}
\usepackage{graphicx}
\usepackage{amsmath,amsfonts,amssymb,amsthm,mathtools} 
\usepackage{pgfplots}
\usepackage{filecontents}
\usepackage{float}
\usepackage{indentfirst}
\usepackage{eucal}
\usepackage{enumitem}
%s\documentclass[openany]{book}
\frenchspacing

\usepackage{indentfirst} % Красная строка

\usetikzlibrary{datavisualization}
\usetikzlibrary{datavisualization.formats.functions}

\usepackage{amsmath}


% Для листинга кода:
\lstset{ %
	language=c,                 % выбор языка для подсветки (здесь это С)
	basicstyle=\small\sffamily, % размер и начертание шрифта для подсветки кода
	numbers=left,               % где поставить нумерацию строк (слева\справа)
	numberstyle=\tiny,           % размер шрифта для номеров строк
	stepnumber=1,                   % размер шага между двумя номерами строк
	numbersep=5pt,                % как далеко отстоят номера строк от подсвечиваемого кода
	showspaces=false,            % показывать или нет пробелы специальными отступами
	showstringspaces=false,      % показывать или нет пробелы в строках
	showtabs=false,             % показывать или нет табуляцию в строках
	frame=single,              % рисовать рамку вокруг кода
	tabsize=2,                 % размер табуляции по умолчанию равен 2 пробелам
	captionpos=t,              % позиция заголовка вверху [t] или внизу [b] 
	breaklines=true,           % автоматически переносить строки (да\нет)
	breakatwhitespace=false, % переносить строки только если есть пробел
	escapeinside={\#*}{*)}   % если нужно добавить комментарии в коде
}


\usepackage[left=2cm,right=2cm, top=2cm,bottom=2cm,bindingoffset=0cm]{geometry}
% Для измененных титулов глав:
\usepackage{titlesec, blindtext, color} % подключаем нужные пакеты
\definecolor{gray75}{gray}{0.75} % определяем цвет
\newcommand{\hsp}{\hspace{20pt}} % длина линии в 20pt
% titleformat определяет стиль
\titleformat{\chapter}[hang]{\Huge\bfseries}{\thechapter\hsp\textcolor{gray75}{|}\hsp}{0pt}{\Huge\bfseries}


% plot
\usepackage{pgfplots}
\usepackage{filecontents}
\usetikzlibrary{datavisualization}
\usetikzlibrary{datavisualization.formats.functions}

\begin{document}
	%\def\chaptername{} % убирает "Глава"
	\thispagestyle{empty}
	\begin{titlepage}
		\noindent \begin{minipage}{0.15\textwidth}
			\includegraphics[width=\linewidth]{b_logo}
		\end{minipage}
		\noindent\begin{minipage}{0.9\textwidth}\centering
			\textbf{Министерство науки и высшего образования Российской Федерации}\\
			\textbf{Федеральное государственное бюджетное образовательное учреждение высшего образования}\\
			\textbf{~~~«Московский государственный технический университет имени Н.Э.~Баумана}\\
			\textbf{(национальный исследовательский университет)»}\\
			\textbf{(МГТУ им. Н.Э.~Баумана)}
		\end{minipage}
		
		\noindent\rule{18cm}{3pt}
		\newline\newline
		\noindent ФАКУЛЬТЕТ $\underline{\text{«Информатика и системы управления»}}$ \newline\newline
		\noindent КАФЕДРА $\underline{\text{«Программное обеспечение ЭВМ и информационные технологии»}}$\newline\newline\newline\newline\newline
		
		\begin{center}
			\noindent\begin{minipage}{1.1\textwidth}\centering
				\Large\textbf{  Отчет по лабораторной работе №6}\newline
				\textbf{по дисциплине <<Функциональное и логическое}\newline
				\textbf{~~~программирование>>}\newline\newline
			\end{minipage}
		\end{center}
		
		\noindent\textbf{Тема} $\underline{\text{Использование функционалов}}$\newline\newline
		\noindent\textbf{Студент} $\underline{\text{Варламова Е. А.~~~~~~~~~~~~~~~~~~~~~~~~~~~~~~~~~~~~~~~~~~~~~~~~~~~~~~~~~~~~}}$\newline\newline
		\noindent\textbf{Группа} $\underline{\text{ИУ7-61Б~~~~~~~~~~~~~~~~~~~~~~~~~~~~~~~~~~~~~~~~~~~~~~~~~~~~~~~~~~~~~~~~~~~~}}$\newline\newline
		\noindent\textbf{Оценка (баллы)} $\underline{\text{~~~~~~~~~~~~~~~~~~~~~~~~~~~~~~~~~~~~~~~~~~~~~~~~~~~~~~~~~~~~~~~~~~~}}$\newline\newline
		\noindent\textbf{Преподаватель} $\underline{\text{Толпинская Н.Б., Строганов Ю. В.~~~~~~~~~~~~~~~~~~~~}}$\newline\newline\newline
		
		\begin{center}
			\vfill
			Москва~---~\the\year
			~г.
		\end{center}
	\end{titlepage}
\setcounter{page}{2}
\section*{Задание 1}
\subsection*{Постановка задачи}
Напишите функцию, которая уменьшает на 10 все числа из списка-аргумента этой
функции.

\begin{lstlisting}
(defun dec (lst) (mapcar #'(lambda (elem) (- elem 10)) lst))
\end{lstlisting}

\section*{Задание №2}
Напишите функцию, которая умножает на заданное число-аргумент все числа
из заданного списка-аргумента.

Все элементы списка --- числа.
\begin{lstlisting}
(defun mul (lst1 num) (mapcar #'(lambda (elem) (* elem num)) lst1))

\end{lstlisting}

Элементы списка -- любые объекты.

\begin{lstlisting}
(defun mul (lst num) 
(mapcar 
#'(lambda (elem) 
(cond ((numberp elem) (* elem num)) 
((listp elem) (mul elem num))  ))
lst)
)
\end{lstlisting}


\section*{Задание №3}
Написать функцию, которая по своему списку-аргументу lst определяет
является ли он палиндромом (то есть равны ли lst и (reverse lst)).
\subsection*{Решение}
\begin{lstlisting}[label=third,caption=Решение задания №3, language=lisp]
(defun to_pairs (lst1 lst2) (mapcar #'equal lst1 lst2))

(defun rdc (lst )(reduce #'(lambda (accum elem) (and accum elem)) lst :initial-value T))

(defun is_same (lst1 lst2) (rdc (to_pairs lst1 lst2)))
	   
(defun is_poly (lst) (is_same lst (reverse lst)))
\end{lstlisting}

\section*{Задание №4}
Написать предикат set-equal, который возвращает t, если два его множества-
аргумента содержат одни и те же элементы, порядок которых не имеет значения.

\subsection*{Решение}
\begin{lstlisting}[label=third,caption=Решение задания №4, language=lisp]
(defun rdc_and (lst )(reduce #'(lambda (accum elem) (and accum elem)) lst :initial-value T))

(defun rdc_or (lst )(reduce #'(lambda (accum elem) (or accum elem)) lst :initial-value nil))

(defun set_equal (l1 l2)
(cond ((eql (length l1) (length l2))
(rdc_and (mapcar #'(lambda (x) 
( rdc_or (mapcar #'(lambda (y) (equal x y)) l2) )
) l1)))
))
\end{lstlisting}

\section*{Задание №5}
Написать функцию которая получает как аргумент список чисел, а возвращает список
квадратов этих чисел в том же порядке

\subsection*{Решение}
\begin{lstlisting}[label=5,caption=Решение задания №5, language=lisp]
(defun sqr (lst)
(mapcar #'(lambda (x) (* x x)) lst)
)
\end{lstlisting}

\section*{Задание №6}
Напишите функцию, select-between, которая из списка-аргумента, содержащего только
числа, выбирает только те, которые расположены между двумя указанными границами-
аргументами и возвращает их в виде списка (упорядоченного по возрастанию списка чисел
(+ 2 балла)).
\subsection*{Решение}
\begin{lstlisting}[label=5,caption=Решение задания №5, language=lisp]
(defun select-between (lst left right) (mapcan #'(lambda (elem) (if (and (< elem right) (> elem left)) (list elem) nil)) lst))

\end{lstlisting}


\section*{Задание №7}
Написать функцию, вычисляющую декартово произведение двух своих списков-
аргументов. ( Напомним, что А х В это множество всевозможных пар (a b), где а
принадлежит А, принадлежит В.)
\subsection*{Решение}
\begin{lstlisting}[label=5,caption=Решение задания №5, language=lisp]
(defun decart (lstx lsty) (mapcan #'(lambda (x) (mapcar #'(lambda (y) (list x y)) lsty)) lstx))

\end{lstlisting}

\section*{Задание №8}
Почему так реализовано reduce, в чем причина?
\begin{lstlisting}[label=5,caption=Решение задания №5, language=lisp]
(reduce #'+0) -> 0 
(reduce #'+ ()) -> 0
\end{lstlisting}

\section*{Задание №9}
Пусть list-of-list список, состоящий из списков. Написать функцию, которая вычисляет сумму длин всех элементов list-of-list, т.е. например для аргумента ((1 2) (3 4)) -> 4.
\subsection*{Решение}
\begin{lstlisting}[label=5,caption=Решение задания №5, language=lisp]
(defun sum-sizes (lst) 
(reduce #'(lambda (accum elem) (+ accum (length elem))) lst :initial-value 0)
)

\end{lstlisting}	
	
\end{document}