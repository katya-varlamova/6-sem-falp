\documentclass[12pt]{report}
\usepackage[utf8]{inputenc}
\usepackage[russian]{babel}
%\usepackage[14pt]{extsizes}
\usepackage{listings}
\usepackage{graphicx}
\usepackage{amsmath,amsfonts,amssymb,amsthm,mathtools} 
\usepackage{pgfplots}
\usepackage{filecontents}
\usepackage{float}
\usepackage{indentfirst}
\usepackage{eucal}
\usepackage{enumitem}
%s\documentclass[openany]{book}
\frenchspacing

\usepackage{indentfirst} % Красная строка

\usetikzlibrary{datavisualization}
\usetikzlibrary{datavisualization.formats.functions}

\usepackage{amsmath}


% Для листинга кода:
\lstset{ %
	language=c,                 % выбор языка для подсветки (здесь это С)
	basicstyle=\small\sffamily, % размер и начертание шрифта для подсветки кода
	numbers=left,               % где поставить нумерацию строк (слева\справа)
	numberstyle=\tiny,           % размер шрифта для номеров строк
	stepnumber=1,                   % размер шага между двумя номерами строк
	numbersep=5pt,                % как далеко отстоят номера строк от подсвечиваемого кода
	showspaces=false,            % показывать или нет пробелы специальными отступами
	showstringspaces=false,      % показывать или нет пробелы в строках
	showtabs=false,             % показывать или нет табуляцию в строках
	frame=single,              % рисовать рамку вокруг кода
	tabsize=2,                 % размер табуляции по умолчанию равен 2 пробелам
	captionpos=t,              % позиция заголовка вверху [t] или внизу [b] 
	breaklines=true,           % автоматически переносить строки (да\нет)
	breakatwhitespace=false, % переносить строки только если есть пробел
	escapeinside={\#*}{*)}   % если нужно добавить комментарии в коде
}


\usepackage[left=2cm,right=2cm, top=2cm,bottom=2cm,bindingoffset=0cm]{geometry}
% Для измененных титулов глав:
\usepackage{titlesec, blindtext, color} % подключаем нужные пакеты
\definecolor{gray75}{gray}{0.75} % определяем цвет
\newcommand{\hsp}{\hspace{20pt}} % длина линии в 20pt
% titleformat определяет стиль
\titleformat{\chapter}[hang]{\Huge\bfseries}{\thechapter\hsp\textcolor{gray75}{|}\hsp}{0pt}{\Huge\bfseries}


% plot
\usepackage{pgfplots}
\usepackage{filecontents}
\usetikzlibrary{datavisualization}
\usetikzlibrary{datavisualization.formats.functions}

\begin{document}
	%\def\chaptername{} % убирает "Глава"
	\thispagestyle{empty}
	\begin{titlepage}
		\noindent \begin{minipage}{0.15\textwidth}
			\includegraphics[width=\linewidth]{b_logo}
		\end{minipage}
		\noindent\begin{minipage}{0.9\textwidth}\centering
			\textbf{Министерство науки и высшего образования Российской Федерации}\\
			\textbf{Федеральное государственное бюджетное образовательное учреждение высшего образования}\\
			\textbf{~~~«Московский государственный технический университет имени Н.Э.~Баумана}\\
			\textbf{(национальный исследовательский университет)»}\\
			\textbf{(МГТУ им. Н.Э.~Баумана)}
		\end{minipage}
		
		\noindent\rule{18cm}{3pt}
		\newline\newline
		\noindent ФАКУЛЬТЕТ $\underline{\text{«Информатика и системы управления»}}$ \newline\newline
		\noindent КАФЕДРА $\underline{\text{«Программное обеспечение ЭВМ и информационные технологии»}}$\newline\newline\newline\newline\newline
		
		\begin{center}
			\noindent\begin{minipage}{1.1\textwidth}\centering
				\Large\textbf{  Отчет по лабораторной работе №2}\newline
				\textbf{по дисциплине <<Функциональное и логическое}\newline
				\textbf{~~~программирование>>}\newline\newline
			\end{minipage}
		\end{center}
		
		\noindent\textbf{Тема} $\underline{\text{Определение функций пользователя~~~~}}$\newline\newline
		\noindent\textbf{Студент} $\underline{\text{Варламова Е. А.~~~~~~~~~~~~~~~~~~~~~~~~~~}}$\newline\newline
		\noindent\textbf{Группа} $\underline{\text{ИУ7-61Б~~~~~~~~~~~~~~~~~~~~~~~~~~~~~~~~~~}}$\newline\newline
		\noindent\textbf{Оценка (баллы)} $\underline{\text{~~~~~~~~~~~~~~~~~~~~~~~~~~~~~~~~~}}$\newline\newline
		\noindent\textbf{Преподаватель} $\underline{\text{Толпинская Н.Б.~~~~~~~~~~~}}$\newline\newline\newline
		
		\begin{center}
			\vfill
			Москва~---~\the\year
			~г.
		\end{center}
	\end{titlepage}
	
\setcounter{page}{2}
\section*{Задание 1}
Составить диаграмму вычисления следующих выражений:

\begin{enumerate}
	\item (equal 3 (abs -3))
	\item (equal (+ 1 2) 3)
	\item (equal (* 4 7) 21)
	\item (equal (* 2 3) (+ 7 2))
	\item (equal (- 7 3) (* 3 2))
	\item (equal (abs (- 2 4)) 3))
\end{enumerate}

\subsection*{Решение}
Диаграммы оформлены на тетрадном листке бумаге и прилагаются к отчёту.

\section*{Задание 2}
Написать функцию, вычисляющую гипотенузу прямоугольного треугольника по заданным катетам и составить диаграмму её вычисления.
\subsection*{Решение}
\begin{lstlisting}[label=second,caption=Решение задания №2, language=lisp]
(defun hypot (a b) (sqrt (+ (* a a) (* b b))))
\end{lstlisting}
Диаграмма оформлена на тетрадном листке бумаге и прилагается к отчёту.

\section*{Задание 3}
Написать функцию, вычисляющую объем параллелепипеда по 3-м его сторонам, и составить диаграмму ее вычисления.

\subsection*{Решение}
\begin{lstlisting}[label=third,caption=Решение задания №3, language=lisp]
(defun v (a b c) (* a b c))
\end{lstlisting}
Диаграмма оформлена на тетрадном листке бумаге и прилагается к отчёту.

\section*{Задание 4}
Каковы результаты вычисления следующих выражений?(объяснить возможную ошибку и варианты ее устранения)
\subsection*{Решение}

\begin{lstlisting}[label=4xd,caption=Решение задания №4, language=lisp]
(list 'a c) -> C IS UNBOUND; (list 'a 'c) -> (A C)
(cons 'a (b c)); undefined function b, C IS UNBOUND; 
(cons 'a '(b c)) -> (A B C)
(caddy (1 2 3 4 5)) -> illegal function call; (caddr (1 2 3 4 5)) -> 3
(cons 'a 'b 'c); INVALID NUMBER OF ARGUMENTS; (cons 'a '(b c)) -> (a b c)
(list 'a (b c));  undefined function b, C IS UNBOUND; (list 'a '(b c)) -> (A (B C))
(list a '(b c)); A IS UNBOUND; (list 'a '(b c)) -> (A (B C))
(list (+ 1 '(length '(1 2 3)))) ; (LENGTH '(1 2 3)) is not of type NUMBER; (list (+ 1 (length '(1 2 3)))) -> (4)
\end{lstlisting}

\section*{Задание 5}
Написать функцию longer\_then от двух списков-аргументов, которая возвращает Т, если первый аргумент имеет большую длину.

\subsection*{Решение}

\begin{lstlisting}[label=5xd,caption=Решение задания №5, language=lisp]
(defun longer_than (l1 l2) (> (length l1) (length l2)))
\end{lstlisting}

\section*{Задание 6}
\subsection*{Постановка задачи}
Каковы результаты вычисления следующих выражений?
\subsection*{Решение}

\begin{lstlisting}[label=6xd,caption=Решение задания №6, language=lisp]
(cons 3 (list 5 6)) -> (3 5 6)
(cons 3 '(list 5 6)) -> (3 LIST 5 6)
(list 3 'from 9 'lives (- 9 3)) -> (3 FROM 9 GIVES 6)
(+ (length for 2 too)) (car '(21 22 23))) -> FOR is unbound; (+ (length '(for 2 too)) (car '(21 22 23))) -> 24
(cdr '(cons is short for ans)) -> (IS SHORT FOR ANS)
(car (list one two)) -> one, two are unbound; (car (list 'one 'two)); -> one
\end{lstlisting}

\section*{Задание 7}
Дана функция (defun mystery (x) (list (second x) (first x))). Какие результаты вычисления следующих выражений?

\subsection*{Решение}
\begin{lstlisting}[label=7xd,caption=Решение задания №7, language=lisp]
(mystery (one two)) -> two is unbound; (mystery '(one two)) -> (TWO ONE)
(mystery one 'two)) -> ONE is unbound 
(mystery (last `one `two)) -> ONE is unbound;
(mystery free) -> FREE is unbound; (mystery `(free)) -> (NIL FREE)
\end{lstlisting}

\section*{Задание 8}
Написать функцию, которая переводит температуру в системе Фаренгейта температуру по Цельсию (defum f-to-c (temp)...).
Формулы: c = 5/9*(f-320); f= 9/5*c+32.0.
Как бы назывался роман Р.Брэдбери "+451 по Фаренгейту" в системе по Цельсию?

\subsection*{Решение}
\begin{lstlisting}[label=8xd,caption=Решение задания №8, language=lisp]
(defun f-to-c (temp) ( * 5/9 (- temp 320)))
\end{lstlisting}

Роман бы назывался "+73 по Цельсию" (с учётом округления числа 655/9).

\section*{Задание 9}
Что получится при вычисления каждого из выражений?

\subsection*{Решение}
\begin{lstlisting}[label=9xd,caption=Решение задания №9, language=lisp]
(list 'cons t NIL) -> (cons T NIL)
(eval (list 'cons t NIL)) -> (eval (cons T NIL)) -> (T)
(eval (eval (list 'cons t NIL))) -> (eval (T)) -> undefined function T
(apply #cons "(t NIL)) -> illegal complex number format: #C~S; (apply #'cons '(t NIL)) -> (T)
(eval NIL) -> NIL
(list 'eval NIL) -> (eval NIL)
(eval (list 'eval NIL)) -> NIL
\end{lstlisting}

\section*{Дополнительно}
\subsection*{Постановка задачи}
1. Написать функцию, вычисляющую катет по заданной гипотенузе и другому катету прямоугольного треугольника, и составить диаграмму ее вычисления.

2. Написать функцию, вычисляющую площадь трапеции по ее основаниям и
высоте, и составить диаграмму ее вычисления.
\subsection*{Решение}
\begin{lstlisting}[label=101xd,caption=Решение задания №10.1, language=lisp]
(defun cat (c h) (sqrt (- (* h h) (* c c))))
\end{lstlisting}
\begin{lstlisting}[label=102xd,caption=Решение задания №10.2, language=lisp]
(defun s (a b h) (* 1/2 h (+ a b)))
\end{lstlisting}
Диаграммы оформлены на тетрадном листке бумаге и прилагаются к отчёту.
\section*{Контрольные вопросы}

\textbf{Вопрос 1.} Базис языка Lisp. \newline
\indent\textbf{Ответ. }
Базис языка образуют атомы, структуры (точечные пары и списки), базовые функции, базовые функционалы (функции, аргументами и значением которых являются функции).

\textbf{Вопрос 2.} Классификация функций языка Lisp.
	
\textbf{Ответ.} 
	
\begin{itemize}
	\item чистые (с фиксированным количеством аргументов) математические функции;
	\item рекурсивные функции;
	\item специальные функции – формы (принимают произвольное количество аргументов);
	\item псевдофункции (создающие «эффект», например, на экране);
	\item функционалы.
\end{itemize}

\textbf{Вопрос 3.} Способы создания функций.
	
\textbf{Ответ.} 
Функцию можно определить с помощью \textbf{defun} или \textbf{lambda.} (defun имя\_функции (список\_аргументов) тело\_функции).

\textbf{Вопрос 4.} Функции \textbf{car}, \textbf{cdr}.
	
\textbf{Ответ.} Функции $car$, $cdr$ являются базовыми функциями доступа к данным. 

$car$ принимает точечную пару или список в качестве аргумента и возвращает первый элемент (если список пустой, то $Nil$). 

$cdr$ принимает точечную пару или список в качестве аргумента и возвращает все элементы, кроме первого или $Nil$.""\newline
	
\textbf{Вопрос 5.} Назначение и отличие в работе \textbf{cons} и \textbf{list}.
	
\textbf{Ответ.} Функции $list$, $cons$ являются функциями создания списков
($cons$ – базовая, $list$ – нет). $cons$ создает списочную ячейку и устанавливает два указателя на аргументы. $list$ принимает переменное число аргументов и возвращает список, элементы которого – переданные в функцию аргументы.
	
\bibliographystyle{utf8gost705u}  % стилевой файл для оформления по ГОСТу
	
\bibliography{51-biblio}          % имя библиографической базы (bib-файла)
	
	
\end{document}